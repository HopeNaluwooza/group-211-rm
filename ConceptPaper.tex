% !TEX TS-program = pdflatex
% !TEX encoding = UTF-8 Unicode

% This is a simple template for a LaTeX document using the "article" class.
% See "book", "report", "letter" for other types of document.

\documentclass[11pt]{article} % use larger type; default would be 10pt

\usepackage[utf8]{inputenc} % set input encoding (not needed with XeLaTeX)

%%% Examples of Article customizations
% These packages are optional, depending whether you want the features they provide.
% See the LaTeX Companion or other references for full information

%%% PAGE DIMENSIONS
\usepackage{geometry} % to change the page dimensions
\geometry{a4paper} % or letterpaper (US) or a5paper or....
% \geometry{margin=2in} % for example, change the margins to 2 inches all round
% \geometry{landscape} % set up the page for landscape
%   read geometry.pdf for detailed page layout information

\usepackage{graphicx} % support the \includegraphics command and options

% \usepackage[parfill]{parskip} % Activate to begin paragraphs with an empty line rather than an indent

%%% PACKAGES
\usepackage{booktabs} % for much better looking tables
\usepackage{array} % for better arrays (eg matrices) in maths
\usepackage{paralist} % very flexible & customisable lists (eg. enumerate/itemize, etc.)
\usepackage{verbatim} % adds environment for commenting out blocks of text & for better verbatim
\usepackage{subfig} % make it possible to include more than one captioned figure/table in a single float
% These packages are all incorporated in the memoir class to one degree or another...

%%% HEADERS & FOOTERS
\usepackage{fancyhdr} % This should be set AFTER setting up the page geometry
\pagestyle{fancy} % options: empty , plain , fancy
\renewcommand{\headrulewidth}{0pt} % customise the layout...
\lhead{}\chead{}\rhead{}
\lfoot{}\cfoot{\thepage}\rfoot{}

%%% SECTION TITLE APPEARANCE
\usepackage{sectsty}
\allsectionsfont{\sffamily\mdseries\upshape} % (See the fntguide.pdf for font help)
% (This matches ConTeXt defaults)

%%% ToC (table of contents) APPEARANCE
\usepackage[nottoc,notlof,notlot]{tocbibind} % Put the bibliography in the ToC
\usepackage[titles,subfigure]{tocloft} % Alter the style of the Table of Contents
\renewcommand{\cftsecfont}{\rmfamily\mdseries\upshape}
\renewcommand{\cftsecpagefont}{\rmfamily\mdseries\upshape} % No bold!

%%% END Article customizations

%%% The "real" document content comes below...

\title{COMPUTING PRICES IN A GENERALIZATION THE PRODUCT–MIX AUCTION}
\author{The Author}
%\date{} % Activate to display a given date or no date (if empty),
         % otherwise the current date is printed 

\begin{document}
\maketitle

\section{Introduction}

The “Product-Mix Auction” is a single-round auction that can be used whenever an auctioneer wants to sell or buy multiple differentiated goods. It allows all participants to express their preferences between varieties, as well as for alternative quantities of specific varieties.
The relevance of this research is to help organizations to acquire products they might need or to sell those they might deem disposable. The findings of this research will enable these organizations to price various products appropriately depending on the reigning market conditions. The target market for this reserach is organizations that at some point in time might need to dispose off company property through product-mix auction.

\section{Background.}
The Product-Mix Auction was devised by Klemperer in 2008 for the purpose of providing liquidity to commercial banks during the financial crisis; it was used for a number of years by the Bank of England.
It uses bids that represent the amounts a buyer is willing to pay for certain bundles of goods, and the "correct" prices (causing the available supply of goods to be released to the buyers) can be found by solving a linear program. The project investigates an extension to the original auction that allows buyers more flexibility to express their requirements. 
This extension, allows "negative bids" to be made, allows a buyer to express any "strong substitutes" demand function. In this extension, the search for prices has the form of a convex optimisation problem, and the project envisages implementing the Ellipsoid Method to find the prices, along with local search heuristics. The Ellipsoid Method runs in polynomial time but is slow in practice; at present we have no guarantee for the performance of the local-search approach, but it's of interest to see how it performs in practice, on simulated data.

\section{Abstract}
The product-mix auction is a fairly new auction type that was adopted by the bank of england. 
Bidders express their relative perferances between varieties as well as alternative quantites of the specfic varieties in simultaneous sealed bids.
 All the information from the bids is then used in conjunction with the auctioneers own perferences. This paper is to investigate how the product prices are computed or set.
 
\section{Project Description.}
This project is intended to find a  suitable way of computing prices in a product-mixed auction. The best approach to this being computation of the minimum price of each item as affected by the sellers, autioneers and bidder perferences.


 \subsection{Goals and objectives} 
 The project is aiming at creating a  much faster and accurate way of prices setting in a product-mix auction and find more satifying appoarch or algorithm to pricing during a produt-mix auction considering all or any factors that affect the product valution .
 
\section{literature review}
early history, the product mix was formulated as an LP problem and applied in a wide variety of settings.some early industry applications included charnes et al(1952) in petro chemical processing, fabian(1958) in integrated iron and steel production, eisemannn and young(1960) in textile manufacturing,koennigsberg(1961) in wood products manufucturing, and swanson and woodruff(1964) in agriculture. over time, formulation innovations emerged to address special features of the decision making enviromen. they included fuzzy features such as the decision makers abililty/ inability to rationalise the tradeoffs in product mix decision-making. recent contributions to accommodate fuzzy modeling aspects included Bhattacharya and vast (2007).review of TOC
also chaharooghi and jahari (2007) presented a review of TOC literature related to product mix determination.\\
The product-mix auction is an intriguing form of action that was devleoped in 2008, Paul Klemperer gives an extensive description of how the whole process operates in his a paper he wrote for the Journal of the European Economic Association, 2010, The most recent public version of this paper is available at http://www.paulklemperer.org. 
He continues to descible how the bids work in the auctions and how the prices are computed. This research paper intends to find a better way to the way the prices set during the action.
Comparing the product mix action to the simultaneous multiple-round auction (SMRA) they are alot of similarities but the product-mix action presents advantages that the SMRA does not for example it is much faster, simipler to us and there for much hard to collide.
it allows the auctioneer as well as the bidders, to specify how the relative quantities of the different varieties to be sold should depend on their relative prices.

\section{Problem and Significance}
Valuation of products at the time of auction is a very complex proceedure especially when dealing with items that are not brand new or items whose valuation fluctuates with the reigning market conditions. This tends to costs organizations a significant amount of resources.
Expenditure on evaluation is very undesirable to organizations since a good number of them, at the time of auctioning, are struglling with liquidity.
We feel that given that the minmum price is set by the auctioneer after all bids have been set it my led to negation of bidders that in turn would have raised their bids depending on their desire for the auctioned item, the project is aimimg at find a compromise to this without affecting the nature of the bids and process 
This research project aims to come up with a formular(algorithm) that be generally used by different organizations to compute prices of different products at the time of disposal through prduct-mix auction. This algorithm should be able to do so while exhausting the different factors that might affect the product valuation.

\subsection{Approach/Methodology} 
The methodololyg to this project is very vitual as the computating of prices in the product mix aution seems somewhat tusking and complex. 
Our intended approach on the subject is to closely look at the product-mix aution a  carried out by the back of England and other organisations that have particapated in the auction in the past this will give us a better understanding of the process. 
The other approach will be to carry out a small mock product-mix Auction trial which will help us better understand the feild nad in turn help us in noting all and any constraints to create a mathematical computation to set prices that best suit the sellers and bidders perferences.


\subsection{Impact and Outcomes}
At the end of this project organisations should be able to sell of products not needed by them anymore in a fast way using product-mix aution, which should be satisfactory both to the buyer abd the bider of the product. Here the organisation gets to sell its products at the price they wanted and the bider gets to walk away with the product they needed.
\section{Project Team}

we worked in a group of three, each one of us had roles assigned,we had several meeting to discuss about the project. 
HopeNaluwooza
OscarTalemwa
JamesIgaba
\section{References}
Ausubel, Lawrence, and Peter Cramton (2008). “A Troubled Asset Reverse
Auction.” mimeo, University of Maryland.\\
Klemperer, Paul (1999). “Auction Theory.” Journal of Economic Surveys, 13
(2), 227-86.\\
Journal of the European Economic Association, 2010, 8, forthcoming
(first version, 2008)\\
http://www.nuff.ox.ac.uk/users/klemperer/productmix.pdf\\



\end{document}
